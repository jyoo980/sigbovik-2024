%%% Some sections of this file are only needed for papers in ACM-sponsored
%%% conferences, not IEEE-sponsored conferences (such as ICSE, every two
%%% years).
%%%
%%% You may be better off using most of this paper template repository, but
%%% mostly replacing this main file by that in the conference's author kit
%%% (copying over the content in this template that is specific to your
%%% paper, such as the \input and \bibliography directives, etc.).

\documentclass[sigconf]{acmart}
\settopmatter{printfolios=false,printccs=false,printacmref=false}

%%
%% \BibTeX command to typeset BibTeX logo in the docs
\AtBeginDocument{%
  \providecommand\BibTeX{{%
    \normalfont B\kern-0.5em{\scshape i\kern-0.25em b}\kern-0.8em\TeX}}}

%% Rights management information.  This information is sent to you
%% when you complete the rights form.  These commands have SAMPLE
%% values in them; it is your responsibility as an author to replace
%% the commands and values with those provided to you when you
%% complete the rights form.
% \setcopyright{acmcopyright}
% \copyrightyear{2018}
% \acmYear{2018}
% \acmDOI{10.1145/1122445.1122456}

%% These commands are for a PROCEEDINGS abstract or paper.
% \acmConference[Woodstock '18]{Woodstock '18: ACM Symposium on Neural
%   Gaze Detection}{June 03--05, 2018}{Woodstock, NY}
% \acmBooktitle{Woodstock '18: ACM Symposium on Neural Gaze Detection,
%   June 03--05, 2018, Woodstock, NY}
% \acmPrice{15.00}
% \acmISBN{978-1-4503-9999-9/18/06}
\acmConference[SIGBOVIK 2024]{18th Conference of the ACH Special Interest Group on Harry Quadratosquamosal Bovik}{4 April, 2024}{Pittsburgh, USA}


%%
%% Submission ID.
%% Use this when submitting an article to a sponsored event. You'll
%% receive a unique submission ID from the organizers
%% of the event, and this ID should be used as the parameter to this command.
%%\acmSubmissionID{123-A56-BU3}

%%
%% The majority of ACM publications use numbered citations and
%% references.  The command \citestyle{authoryear} switches to the
%% "author year" style.
%%
%% If you are preparing content for an event
%% sponsored by ACM SIGGRAPH, you must use the "author year" style of
%% citations and references.
%% Uncommenting
%% the next command will enable that style.
%%\citestyle{acmauthoryear}

\author{James Yoo}
\affiliation{%
  \institution{University of Washington}
  \city{Seattle}
  \state{Washington}
  \country{USA}
}
\email{jmsy@cs.washington.edu}

% tables
\usepackage{tabularx}
\usepackage{booktabs}

%\usepackage{algorithm}
%\usepackage[noend]{algpseudocode}

% xspace command
\usepackage{xspace}

% for the researchquestions environment
\usepackage{enumitem}

% From https://tex.stackexchange.com/questions/177025/
%% \makeatletter
%% \newcounter{algorithmicH}% New algorithmic-like hyperref counter
%% \let\oldalgorithmic\algorithmic
%% \renewcommand{\algorithmic}{%
%%   \stepcounter{algorithmicH}% Step counter
%%   \oldalgorithmic}% Do what was always done with algorithmic environment
%% \renewcommand{\theHALG@line}{ALG@line.\thealgorithmicH.\arabic{ALG@line}}
%% \makeatother

% lstlisting command
\usepackage{listings}
\usepackage[scaled]{beramono}
\newcommand*\LSTfont{\Small\fontencoding{T1}\ttfamily\SetTracking{encoding=*}{-60}\lsstyle}
\lstset{language=Java,
  frame=none,
  aboveskip=1.5pt,
  belowskip=0pt,
  showstringspaces=false,
  columns=flexible,
  basicstyle=\LSTfont,
  numbers=none,
  numberstyle=\tiny\color{black},
  keywordstyle=\color{black},
  commentstyle=\color{black},
  stringstyle=\color{black},
  breaklines=true,
  breakatwhitespace=true,
  tabsize=3,
  %emph={@NonNegative,@Positive,@GTENegativeOne,@LTLengthOf,@LTEqLengthOf,@IndexFor,@IndexOrHigh,@IndexOrLow,@HasSubsequence,@LessThan,@SameLen,@SearchIndexFor,@MinLen,@ArrayLen,@IntVal,@IntRange,@LengthOf,@UpperBoundUnknown,@LowerBoundUnknown,int,double,List,Map,Object,SerialDate,Long,Integer,DefaultPolarItemRenderer,LegendItem,PolarPlot,XYDataset,long,T,String,string,byte,InputStream,CategoryDataset,DatasetRenderingOrder,ArrayList,Entry,Values,Number,ValuesContract,ImmutableIntArray,Dataset,XYZDataset}, emphstyle=\color{blue}
}

% Graphics
\usepackage{tikz}
\usetikzlibrary{arrows,automata,positioning}
\usepackage{standalone}

% Change font and line spacing for figure captions
\usepackage{setspace,caption}
\captionsetup{labelfont={small,bf}, textfont={small,bf,stretch=0.8}, labelsep=colon, margin=0pt}

\usepackage{flushend} % balanced columns on last page

\usepackage{url} % URLs; used in plume-bib

% cref command; best to load last
\usepackage{cleveref}
\newcommand{\crefrangeconjunction}{--}

%%
%% end of the preamble, start of the body of the document source.
\begin{document}

% Use a macro like this for your tool's name. Do not use your tool's name
% in the text: always use the macro.
\newcommand{\theTypeInferenceFramework}{the Type Inference Framework\xspace}
\newcommand{\TheTypeInferenceFramework}{The Type Inference Framework\xspace}
\newcommand{\TIF}{TIF\xspace}
%% Alternate formulation; use whichever of the two you prefer:
% \newcommand{\tool}{the Type Inference Framework\xspace}
% \newcommand{\Tool}{The Type Inference Framework\xspace}
% \newcommand{\toolShort}{TIF\xspace}

\newcommand{\conferencePageLimit}{10}

%%% Todo comments
\newcommand{\todo}[1]{{\color{red}\bfseries [[#1]]}}
%% Comment or uncomment this line.
%\renewcommand{\todo}[1]{\relax}

\newcommand{\manu}[1]{\todo{#1 --MS}}

% Don't show todo commands if this macro is defined.
\ifdefined\notodocomments
  \renewcommand{\todo}[1]{\relax}
\fi

% Use like: \ifanonymous{ANONYMOUS TEXT}\else{NON-ANONNYMOUS TEXT}\fi
% where the "\else{NON-ANONNYMOUS TEXT}" may be omitted.
\newif\ifanonymous
%% Comment or uncomment this line
\anonymoustrue

\newcommand{\anonurl}[1]{\ifanonymous URL removed for anonymity.\else\url{#1}\fi}
\newcommand{\footnoteanonurl}[1]{\footnote{\anonurl{#1}}}

% \|name| or \mathid{name} denotes identifiers and slots in formulas
\def\|#1|{\mathid{#1}}
\newcommand{\mathid}[1]{\ensuremath{\mathit{#1}}}
% \<name> or \codeid{name} denotes computer code identifiers
\def\<#1>{\codeid{#1}}
% \protected\def\codeid#1{\ifmmode{\mbox{\sf{#1}}}\else{\sf #1}\fi}
% \protected\def\codeid#1{\ifmmode{\mbox{\ttfamily{#1}}}\else{\ttfamily #1}\fi}
\protected\def\codeid#1{\ifmmode{\mbox{\smaller\ttfamily{#1}}}\else{\smaller\ttfamily #1}\fi}
% \protected\def\codeid#1{\mintinline{java}{#1}}

% research question list, based on the answer to https://tex.stackexchange.com/questions/559305/how-to-format-for-two-column-research-question
\newlist{researchquestions}{enumerate}{1}
\setlist[researchquestions]{label*=\textbf{RQ\arabic*}}

\newcommand{\CalledMethodsBottom}{\<@Call\-ed\-Meth\-ods\-Bottom>\xspace}
\newcommand{\CalledMethods}{\<@Call\-ed\-Meth\-ods>\xspace}
\newcommand{\EnsuresCalledMethods}{\<@En\-sures\-Call\-ed\-Meth\-ods>\xspace}
\newcommand{\MustCall}{\codeid{@Must\-Call}\xspace}
\newcommand{\MustCallAlias}{\codeid{@Must\-Call\-Alias}\xspace}
\newcommand{\MustCallUnknown}{\codeid{@Must\-Call\-Unknown}\xspace}
\newcommand{\CreatesMustCallFor}{\<@Creates\-Must\-Call\-For>\xspace}
% Deprecated
\newcommand{\ResetMustCall}{\CreatesMustCallFor}

% "trule" stands for ``type rule''
\newcommand{\trule}[2]{\[\frac{#1}{#2}\]}
\newcommand{\truleinline}[2]{\ensuremath{#1\mathrel{\vdash}#2}}
\newcommand{\hastype}[1]{\mathbin{:}\trtext{#1}}
\newcommand{\trcode}[1]{\codeid{\smaller\smaller #1}}
\newcommand{\trtext}[1]{\mbox{\smaller\smaller #1}}
\newcommand{\trquoted}[1]{\trcode{"}#1\trcode{"}}


%%% Computed values

% For any number that's referenced in the text itself (and a table), create a macro like these rather than copy-pasting.
% These examples are from the WPI paper; you can delete them.

\newcommand{\numTypeSystems}{11\xspace} % Formatter, index, interning, lock, nullness, regex, resourceleak, signature, signedness, InitializedFields, Optional
\newcommand{\numModifiedTypeSystems}{2\xspace} % Formatter, Nullness (not Called Methods, because the postcondition code is general)
\newcommand{\numProjects}{12\xspace}
\newcommand{\numLOC}{88,680\xspace}
\newcommand{\numHumanAnnos}{803\xspace}
\newcommand{\percentInferred}{39\todo{check}\%\xspace}
\newcommand{\warningReductionPercent}{45\todo{check}\%\xspace}
\newcommand{\tsSpecificLoC}{61\todo{check}\xspace}

%%% Miscellaneous

\hyphenation{type-state}        % LaTeX defaults to "types-tate"
\hyphenation{null-able}         % LaTeX defaults to "nul-lable"

%%% Space-saving hacks

% Reduce indentation in lists.
\setlength{\leftmargini}{.75\leftmargini}
\setlength{\leftmarginii}{.75\leftmarginii}
\setlength{\leftmarginiii}{.75\leftmarginiii}

\newcommand{\prefigcaption}{\vspace{-5pt}}
\newcommand{\posttablecaption}{\vspace{-5pt}}

% Reduce the separation between figures and text.
\addtolength{\textfloatsep}{-.25\textfloatsep}
\addtolength{\dbltextfloatsep}{-.25\dbltextfloatsep}
\addtolength{\floatsep}{-.25\floatsep}
\addtolength{\dblfloatsep}{-.25\dblfloatsep}

\newcommand{\zph}{\phantom{0}}
\newcommand{\zzph}{\phantom{00}}

\newcommand{\ie}{i.e.,\xspace}
\newcommand{\eg}{e.g.,\xspace}


%%
%% The "title" command has an optional parameter,
%% allowing the author to define a "short title" to be used in page headers.
\title{Defeating Rickrolls via Pluggable Types}

%%
%% The "author" command and its associated commands are used to define
%% the authors and their affiliations.
%% Of note is the shared affiliation of the first two authors, and the
%% "authornote" and "authornotemark" commands
%% used to denote shared contribution to the research.


%%
%% By default, the full list of authors will be used in the page
%% headers. Often, this list is too long, and will overlap
%% other information printed in the page headers. This command allows
%% the author to define a more concise list
%% of authors' names for this purpose.
%\renewcommand{\shortauthors}{Trovato and Tobin, et al.}

%%
%% The abstract is a short summary of the work to be presented in the
%% article.
\begin{abstract}
\todo{This is an example paper outline.  It briefly describes the
  problem, the key ideas of the solution, and the experimental results.}

\todo{For more advice about writing a paper, see
  \url{https://homes.cs.washington.edu/~mernst/advice/write-technical-paper.html}.}


\label{dummy-label-for-etags:abstract}

\end{abstract}

%%
%% The code below is generated by the tool at http://dl.acm.org/ccs.cfm.
%% Please copy and paste the code instead of the example below.
%%
% \begin{CCSXML}
% <ccs2012>
%  <concept>
%   <concept_id>10010520.10010553.10010562</concept_id>
%   <concept_desc>Computer systems organization~Embedded systems</concept_desc>
%   <concept_significance>500</concept_significance>
%  </concept>
%  <concept>
%   <concept_id>10010520.10010575.10010755</concept_id>
%   <concept_desc>Computer systems organization~Redundancy</concept_desc>
%   <concept_significance>300</concept_significance>
%  </concept>
%  <concept>
%   <concept_id>10010520.10010553.10010554</concept_id>
%   <concept_desc>Computer systems organization~Robotics</concept_desc>
%   <concept_significance>100</concept_significance>
%  </concept>
%  <concept>
%   <concept_id>10003033.10003083.10003095</concept_id>
%   <concept_desc>Networks~Network reliability</concept_desc>
%   <concept_significance>100</concept_significance>
%  </concept>
% </ccs2012>
% \end{CCSXML}

% \ccsdesc[500]{Computer systems organization~Embedded systems}
% \ccsdesc[300]{Computer systems organization~Redundancy}
% \ccsdesc{Computer systems organization~Robotics}
% \ccsdesc[100]{Networks~Network reliability}

%%
%% Keywords. The author(s) should pick words that accurately describe
%% the work being presented. Separate the keywords with commas.
% \keywords{datasets, neural networks, gaze detection, text tagging}

%% A "teaser" image appears between the author and affiliation
%% information and the body of the document, and typically spans the
%% page.
%% \begin{teaserfigure}
%%   \includegraphics[width=\textwidth]{sampleteaser}
%%   \caption{Seattle Mariners at Spring Training, 2010.}
%%   \Description{Enjoying the baseball game from the third-base
%%   seats. Ichiro Suzuki preparing to bat.}
%%   \label{fig:teaser}
%% \end{teaserfigure}

%%
%% This command processes the author and affiliation and title
%% information and builds the first part of the formatted document.
\maketitle

\section{Introduction}
\label{sec:intro}

Rick Astley captured lighting in a bottle when he released his iconic hit
single ``Never Gonna Give You Up''~\cite{NeverGonnaGiveYouUpWiki} on July 27,
1987.
People laughed, people cried, but mostly, people accepted it as among the best
works of musical art ever written.
After its dominance as the zeitgeist of the 90s, ``Never Gonna Give You Up''
faded into relative obscurity until it was revived with a far more sinister
aim.

Today, ``Never Gonna Give You Up'' is weaponized as a deadly tool of mass
tomfoolery; amateur shitposters disguise a link to the YouTube
video~\cite{NeverGonnaGiveYouUpYouTube} in innocuous links, while it is a
mainstay of the toolkits of cybercriminals across the globe.
It is hopeless to mitigate rickroll attacks without the help of automated
software tooling and analyses.
Tooling to detect rickroll attacks should ideally have the following
characteristics:

\begin{itemize}
  \item \textit{Sound}: tools should capture all possible
    instances where a rickroll might be injected into the code.
  \item \textit{Fast}: tools should not introduce additional
    time into the code-build-test development loop.
  \item \textit{Precise}: tools should aim to minimize the false positive
    alarm rate; programmers may ignore output or disable tools altogether
    if the false positive rate is high.
\end{itemize}

\noindent Verifying all non-trivial semantic properties of a a program is
generally undecidable~\cite{Rice53} in practice.
Consequently, it is difficult for a tool to demonstrate all three properties.

In this paper, we propose a type system that may be used to model rickrolls
in a program and show that it is \textit{sound}.
We implement our type system in an open-source formal verification tool for
Java, called the Rickroll Checker.
We show that the Rickroll Checker is \textit{fast} and \textit{precise} in
detecting rickrolls in Java programs.

Our primary contributions are as follows:

\begin{itemize}
  \item A generally-applicable type system for rickrolls that
    provides a compile-time guarantee of the absence of rickrolls at run time.
  \item The Rickroll Checker, an open-source formal verification tool that
    implements our type system for modelling rickrolls.
\end{itemize}


\section{Background: Pluggable Type Systems}
\label{sec:background}

A type system is a conservative approximation of the set of legal values in a
program.
Type-checking is a static analysis over a type system that verifies and
enforces the constraints expressed by types in a program.
For example, a variable declared to hold a string should not be assigned an
integer, and vice-versa.

A \textit{pluggable type-system} enables end-users (\eg{ programmers}) to
extend an existing type system (the host type system) with additional
information.
Programmers may want to use pluggable type-systems when the host type system
cannot capture a relevant problem domain.
For example, in Java, a pluggable type-system may be implemented via
annotations that enable programmers to denote \textit{qualified types}.
A programmer may declare a variable to hold some user input:
\<String userInput = scanner.nextLine();>.
However, there is no guarantee that this user input is sanitized or otherwise
trusted by the system.
In an effort to mitigate errors related to untrusted user input, a programmer
may augment the existing type: \<@Tainted String userInput = scanner.nextLine();>.
The fully-qualified type of \<userInput> is \<@Tainted String>.

A Java compiler plug-in may use this additional information to warn about uses
of the \<userInput> variable across system boundaries, such as querying a
database or making API requests.
For example, the parameter to some database query method may be annotated
as: \<@Untainted String query>, passing \<userInput> to this method would
result in a compile-time type error.
This type of safety would not be enforced by a standard Java compiler that does
not make use of any pluggable type systems.


\section{A Type System for Rickrolls}
\label{sec:rickroll-type-system}

\Cref{fig:rickroll-type-system} presents our type system for rickrolls, which
comprises 3 elements:

\begin{itemize}
  \item \<@MaybeRickroll>: the top type; represents a strings that may or may
    not contain a link to a rickroll.
  \item \<@NotRickroll>: represents strings that definitely do not contain
    links to rickrolls.
  \item \<@RickrollBottom>: the bottom type in the type hierarchy.
\end{itemize}

\begin{figure}[ht!]
    \centering
    \includestandalone{figures/rickroll-type-system}
    \postfigurespace
    \caption{The type hierarchy for rickrolls.}
    \label{fig:rickroll-type-system}
\end{figure}

We have implemented this type system as the Rickroll Checker for the Checker
Framework~\cite{CheckerFrameworkManualNoYear}, a tool for building pluggable
type systems for Java.
We discuss how the type system may be used to extend the type sytem for
\<String>s in Java in \cref{subsec:rickrolls} and \cref{subsec:not-rickrolls}
and discuss additional features of the type system in \cref{subsec:flow-sensitive-type-refinement}.

\subsection{Rickrolls (Maybe)}
\label{subsec:rickrolls}

In our type system, all strings are guilty until proven innocent.
That is, we treat any declaration of \<String> to possibly be a rickroll;
any values of type \<String> are implicitly augmented with
\<@MaybeRickroll>; programmers need not write this type

\prelistingspace
\begin{lstlisting}[frame=tb]
// Implicitly @MaybeRickroll.
String s1 = "https://www.youtube.com/watch?v=dQw4w9WgXcQ";
// Explicitly @MaybeRickroll
@MaybeRickroll String s2 = "Hello, world!";
\end{lstlisting}
\postlistingspace

\noindent This enables type-checking to soundly assume that every string value
in the program may be an attack vector for a rickroll.
Though this appears to an overbearing default inference by the type system,
we have not found this to be the case.
Having the \<@MaybeRickoll> type be the default enables information that may
be used by the type system to forbid calls that may use a string
(\eg{ \<new URI(..)>}) where a programmer has not yet validated that the string
does not contain a rickroll link:

\prelistingspace
\begin{lstlisting}[frame=tb]
String s1 = "https://www.youtube.com/watch?v=dQw4w9WgXcQ";
URI uri = new URI(s1); // Type error.
\end{lstlisting}
\postlistingspace

\noindent The call to \<new URI(s1);> yields a compile-time type error, as...

\subsection{Not Rickrolls}
\label{subsec:not-rickrolls}

\subsection{Flow-sensitive Type Refinment}
\label{subsec:flow-sensitive-type-refinement}


%% \section{Technique}
\label{sec:technique}

\todo{This section or sections explains the new techniques we used to solve the problem.
  Often, you'll want to put multiple sections here: for example, you might have an overview
  followed by a section of proofs---but it will depend on the exact content of the paper.
  Optionally, it may be preceded by a ``Background'' section that gives relevant context:
  for example, in papers about new Checker Framework checkers, we often include a background
  section that explains what type qualifiers and pluggable types are, the
  annotation syntax, etc.}

\todo{Give all sections descriptive names, not something bland like
  ``Technique'' that could appear in any paper.  Make your section names
  work for you.}

%% 
%% \section{Implementation}
\label{sec:impl}

\todo{Any paper that involves building a tool should include an ``implementation'' section.
  It may discuss a new tool you have built and/or the experimental infrastructure.
  Discuss the key non-obvious design decisions and complications.
  You should also mention: 1) what language the tool targets, 2) any important libraries that
  the tool relies on (\eg the Checker Framework), and 3) that the tool is
  open-source.

  Sometimes this section is very short.}

%% 
%% \section{Evaluation}
\label{sec:evaluation}

\todo{All experimental papers need a good evaluation section.
  This section might include a list of research questions.
  Often, there are separate sections for the experimental methodology and
  the results.

  One of the first things that I do when
  drafting the outline for a new paper is to design the ``main table'': that is, I add
  to this section a table that has row and column headers (but not actual numbers). I find this
  forces me to think about what I'm actually planning to measure, which helps design both
  better narratives (for the intro) and better experiments. This section might also be split
  up into multiple sections if there is a logical grouping of the experiments (\eg open- vs.\
  closed-source subjects, comparisions with other tools, etc.).}

\todo{It's common to include ``research questions'' in the structure
  of your evaluation narrative.  You can use the \<\textbackslash
  researchquestions> environment to automatically number and format
  them, like this:
  \begin{researchquestions}
  \item How do you write a good evaluation section?
  \end{researchquestions}
}

%% 
%% \section{Limitations and Threats to Validity}
\label{sec:limitations}

\todo{This section is required in SE papers. You should familiarize yourself
  with the terms ``internal'', ``external'', and ``construct'' threats to
  validity, and be prepared to use them to discuss your experiments in this
  section.

  One description appears at
  \url{https://github.com/mernst/uwisdom/blob/wiki/Research.adoc} (search
  for ``Threats'') but perhaps it can be improved.}

%% 
%% \section{Related Work}
\label{sec:relatedwork}

\todo{A related work section is not just a list of similar papers:
  its goal is to place this paper in context amongst the larger field.
  That means that anything you discuss in this section needs to not
  only be \emph{explained} on its own, but also must be \emph{compared}
  to the work described in this paper. When drafting, I usually start
  with just a list of related papers, and create the real related work
  section towards the end of the process (once the work in this paper has
  stabilized).}

\todo{Write a citation as ``WPI\textasciitilde\textbackslash
cite\{KelloggDNAE2023\}''
(which formats as ``WPI~\cite{KelloggDNAE2023}'')
and not as ``WPI\textbackslash cite\{KelloggmDNAE2023\}''
(which formats as ``WPI\cite{KelloggDNAE2023}'' with the citation jammed
against the preceding word).
That is, precede each use of \<\textbackslash cite> by a non-breaking
space, which is written as ``\<\textasciitilde>'' in a \<.tex> file.}

%% 
%% \section{Conclusion}

\todo{Recap the contributions of the paper.  (In fact, it is good to think of
  this as a ``contributions'' section even if it's titled ``conclusion''.)
  This section can do so in more depth or with more reference to techniques
  and insights than the abstract and introduction were able to do, since
  readers of those sections wouldn't have read the whole paper yet.
  In other cases, this section is brief.}


\todo{The page limit is {\conferencePageLimit} pages.  This is page \thepage.}

% \begin{acks}
% \end{acks}

%% The next two lines define the bibliography style to be used, and
%% the bibliography file.
\bibliographystyle{ACM-Reference-Format}
\bibliography{local,plume-bib/bibstring-abbrev,plume-bib/types,plume-bib/dispatch,plume-bib/ernst,plume-bib/soft-eng,plume-bib/crossrefs}

%%
%% If your work has an appendix, this is the place to put it.

\end{document}
\endinput
%%
