\section{Background: Pluggable Type Systems}
\label{sec:background}

A type system is a conservative approximation of the set of legal values in a
program.
Type-checking is a static analysis over a type system that verifies and
enforces the constraints expressed by types in a program.
For example, a variable declared to hold a string should not be assigned an
integer, and vice-versa.

A \textit{pluggable type-system} enables end-users (\eg{ programmers}) to
extend an existing type system (the host type system) with additional
information.
Programmers may want to use pluggable type-systems when the host type system
cannot capture a relevant problem domain.
For example, in Java, a pluggable type-system may be implemented via
annotations that enable programmers to denote \textit{qualified types}.
A programmer may declare a variable to hold some user input:
\<String userInput = scanner.nextLine();>.
However, there is no guarantee that this user input is sanitized or otherwise
trusted by the system.
In an effort to mitigate errors related to untrusted user input, a programmer
may augment the existing type: \<@Tainted String userInput = scanner.nextLine();>.
The fully-qualified type of \<userInput> is \<@Tainted String>.

A Java compiler plug-in may use this additional information to warn about uses
of the \<userInput> variable across system boundaries, such as querying a
database or making API requests.
For example, the parameter to some database query method may be annotated
as: \<@Untainted String query>, passing \<userInput> to this method would
result in a compile-time type error.
This type of safety would not be enforced by a standard Java compiler that does
not make use of any pluggable type systems.

