\section{A Type System for Rickrolls}
\label{sec:rickroll-type-system}

\Cref{fig:rickroll-type-system} presents our type system for rickrolls, which
comprises 3 elements:

\begin{itemize}
  \item \<@MaybeRickroll>: the top type; represents a strings that may or may
    not contain a link to a rickroll.
  \item \<@NotRickroll>: represents strings that definitely do not contain
    links to rickrolls.
  \item \<@RickrollBottom>: the bottom type in the type hierarchy.
\end{itemize}

\begin{figure}[ht!]
    \centering
    \includestandalone{figures/rickroll-type-system}
    \postfigurespace
    \caption{The type hierarchy for rickrolls.}
    \label{fig:rickroll-type-system}
\end{figure}

We have implemented this type system as the Rickroll Checker for the Checker
Framework~\cite{CheckerFrameworkManualNoYear}, a tool for building pluggable
type systems for Java.
We discuss how the type system may be used to extend the type sytem for
\<String>s in Java in \cref{subsec:rickrolls} and \cref{subsec:not-rickrolls}
and discuss additional features of the type system in \cref{subsec:flow-sensitive-type-refinement}.

\subsection{Rickrolls (Maybe)}
\label{subsec:rickrolls}

In our type system, all strings are guilty until proven innocent.
That is, we treat any declaration of \<String> to possibly be a rickroll;
any values of type \<String> are implicitly augmented with
\<@MaybeRickroll>; programmers need not write this type

\prelistingspace
\begin{lstlisting}[frame=tb]
// Implicitly @MaybeRickroll.
String s1 = "https://www.youtube.com/watch?v=dQw4w9WgXcQ";
// Explicitly @MaybeRickroll
@MaybeRickroll String s2 = "Hello, world!";
\end{lstlisting}
\postlistingspace

\noindent This enables type-checking to soundly assume that every string value
in the program may be an attack vector for a rickroll.
Though this appears to an overbearing default inference by the type system,
we have not found this to be the case.
Having the \<@MaybeRickoll> type be the default enables information that may
be used by the type system to forbid calls that may use a string
(\eg{ \<new URI(..)>}) where a programmer has not yet validated that the string
does not contain a rickroll link:

\prelistingspace
\begin{lstlisting}[frame=tb]
String s1 = "https://www.youtube.com/watch?v=dQw4w9WgXcQ";
URI uri = new URI(s1); // Type error.
\end{lstlisting}
\postlistingspace

\noindent The call to \<new URI(s1);> yields a compile-time type error, as...

\subsection{Not Rickrolls}
\label{subsec:not-rickrolls}

\subsection{Flow-sensitive Type Refinment}
\label{subsec:flow-sensitive-type-refinement}
